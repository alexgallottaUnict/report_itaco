% Distributed System project report
\documentclass[11pt, a4paper, twoside, notitlepage]{report}
\usepackage[italian]{babel} %capitolo, sommario e altro in italiano
\usepackage[Lenny]{fncychap} 
\usepackage{fullpage}
\usepackage{setspace}
\usepackage{graphicx}
\usepackage{amsfonts}
\usepackage{wrapfig}  
\usepackage[footnotesize,bf]{caption} %dimensione caption
\usepackage{appendix}
\usepackage[utf8x]{inputenc} %caratteri accentati per linux

\linespread{1.5}    %spaziatura linee
 
\begin{document}
% Logo università Catania
\begin{figure}[h] \hspace*{130pt}
\includegraphics[scale=0.25]{img/unict-bianco.png}
\end{figure}
\begin{center}
\begin{Large}
Università degli Studi di Catania
\end{Large}
\begin{large}
\\Corso di Laurea Magistrale in Ingegneria Informatica
\end{large}
\vspace{30pt}

\textbf{
\begin{Huge}
ITAco
\end{Huge}
}
\begin{large}
\\Linguaggi e Traduttori
\end{large}
\vspace{30pt}
\\Realizzato da:

\begin{Large}
%\begin{flushright}
Alessandro Gallotta, Luca Marturana, Rosario Villari
%\end{flushright}
\end{Large}
\end{center}
\vspace{30pt}
\begin{center}
Docente
\begin{Large}
\\Prof.ssa V. Carchiolo
\end{Large}

\vspace{20pt}
\vfill
\begin{small}
18 Marzo 2012
\end{small}
\end{center}

\thispagestyle{empty} %togli numerazione pagina
\clearpage
\begin{figure}[h] \hspace*{140pt}
\includegraphics[scale=0.5]{img/itaco_logo.png}
\end{figure}
\begin{abstract}
Questo documento riassume brevemente le peculiarità principali del progetto
ITAco da un punto di vista interessato alla dimostrazione ai fini accademici
delle competenze acquisite dagli autori dello stesso nell'ambito della creazione
di linguaggi e compilatori.
\\Il nome è stato scelto seguendo l'idea iniziale di creare un linguaggio di
programmazione molto semplice e intuitivo in lingua italiana, affinché possa
essere facile da utilizzare e didatticamente proficuo per delle categorie di
utenti che possono incontrare difficoltà nella memorizzazione di parole chiave
in lingua straniera.
\\Con lo scopo di perseguire questo obiettivo sono state prese decisioni quali
l'inversione del normale assegnamento dei valori\footnote{L'assegnazione in
ITAco avviene con l'espressione nella parte sinistra e la variabile che riceve
il valore nella parte destra}, che, sebbene risulti a un primo impatto
sconveniente a un utente abituato ad altri linguaggio di programmazione, è
tuttavia molto più intuitivo per un utente abituato ad utilizzare la logica
\emph{della calcolatrice} dove prima si inseriscono i valori e
successivamente li si assegna a una varabile.
\\Altre caratteristiche con questo obiettivo possono essere lette nella sezione
riguardante la sitnassi\footnote{~\ref{sintassi}}. Nell'ambito della facilità di
utilizzo rientrano le scelte di una intuitiva interfaccia grafica mentre,
nell'ambito didattico, la possibilità di traduttre il programma scritto in
diversi linguaggi\footnote{al momento i linguaggi utilizzati sono Ruby, C e
codice Jasmin}.
Questi argomenti sono descritti rispettivamente nella sezione ~\ref{gui} e al capitolo ~\ref{back_end}.
\\Vengono inoltre descritti anche gli strumenti, sia teorici che
pratici, utilizzati per la realizzazione del progetto quali JFlex, CUP, le produzioni
della grammatica utilizzata, i test automatici.

\begin{flushright}
{\bf Keywords:} Compilatori, Cup, JFlex
\end{flushright}
\end{abstract}

\onehalfspacing

%Inizia il conteggio pagine e stampa table of contents (indice)
\setcounter{page}{1}
\setcounter{tocdepth}{3}%profondità table of contents
\tableofcontents

\chapter{Front End}

\section{Scanner}
\subsection{Token utilizzati} 
\section{Parser}
\subsection{Produzioni utilizzate}

\chapter{Back End}
\label{back_end}
\section{Linguaggio target Jasmin}
\section{Linguaggio target C}
\section{Linguaggio target Ruby}

\chapter{Caratteristiche del Linguaggio}
\section{Sintassi Utilizzata}
\label{sintassi}
Un altro accorgimento è la definizione delle funzioni nella forma

\begin{verbatim}
nome_funzione (arg1 | arg2 | arg3) -> valore di ritorno
\end{verbatim}

\chapter{Altri Strumenti}
In questo capitolo sono elencati degli strumenti che gli autori del progetto
hanno ritenuto opportuno aggiungere per aumentare vuoi l'usabilità e in generale
la qualità del programma finito vuoi la facilità di sviluppo e il bug-tracking.
\section{Test Automatici}
Al fine di velocizzare lo sviluppo e mantenere un alto livello di controllo sui
bug, sono stati introdotti dei test automatici che coprono la gamma di
funzionalità offerte dal linguaggio ITAco. Poiché le specifiche del progetto
richiedevano principalemente la compilazione usando come linguaggio target
\emph{Jasmin}, i test sono stati creati ad hoc per tale linguaggio. La natura di
alto livello degli altri linguaggi target non hanno fatto ritenere al team di
autori che la necessità di creare appositi test superasse il valore del tempo
necessario per crearli.
\\I test funzionano nel seguente modo:
\begin{enumerate}
  \item La classe \emph{JasminTest} compila e carica dei file .ita creati
  appositamente per i test;
  \item la suddetta classe invia determinati valori di input dei quali conosce
  già i risultati;
  \item infine preleva e confronta i risultati ottenuti dal file compilato con
  quelli attesi.
\end{enumerate}

\section{Esempi}
\section{Interfaccia}
\subsection{CLI}
\subsection{GUI}
\label{gui}

\end{document}
